\section*{Appendix}
\label{sec:appendix}

\subsection{Code Repository Structure}

The complete implementation is available with the following directory structure:

\begin{verbatim}
src/
├── domain/                    # Domain layer
│   ├── models/               # Business entities
│   │   └── conversation.py
│   └── services/             # Business logic
│       ├── text_preprocessor.py
│       ├── emotion_detection_service.py
│       └── behavior_pattern_service.py
├── infrastructure/           # Infrastructure layer
│   ├── config/              # Configuration (12-Factor)
│   │   ├── chatbot_settings.py
│   │   ├── qdrant_settings.py
│   │   └── mongodb_settings.py
│   ├── ml/chatbots/         # Chatbot implementations
│   │   ├── aiml_chatbot.py
│   │   ├── dialogpt_chatbot.py
│   │   ├── rag_chatbot.py
│   │   └── enhanced_rag_chatbot.py
│   ├── memory/              # RAG memory services
│   │   ├── rag_memory_service.py
│   │   └── enhanced_prompt_builder.py
│   └── database/            # Database repositories
│       └── patient_repository.py
└── interfaces/              # Interface layer
    └── cli/
        └── rag_chatbot_cli.py

data/knowledge_bases/aiml/   # AIML rule files
├── therapy.aiml             # Mental health patterns
└── general.aiml             # General conversation

tests/                       # Test suite
├── unit/domain/
│   └── test_text_preprocessor.py
└── conftest.py
\end{verbatim}

\subsection{Configuration Example}

\textbf{.env file configuration}:

\begin{verbatim}
# Gemini API Configuration
GEMINI_API_KEY=your_api_key_here
GOOGLE_API_KEY=alternative_key

# Qdrant Vector Database Configuration
QDRANT_HOST=localhost
QDRANT_PORT=6333
QDRANT_COLLECTION=chatbot_memory

# MongoDB Configuration
MONGODB_URI=mongodb://localhost:27017/
MONGODB_DATABASE=chatbot_patients

# Chatbot Settings
MODEL_NAME=gemini-2.0-flash-exp
TEMPERATURE=0.7
MAX_TOKENS=1000
TOP_P=0.9
\end{verbatim}

\subsection{Running the Chatbots}

\textbf{AIML Chatbot}:
\begin{verbatim}
python -m src.interfaces.cli.rag_chatbot_cli \\
    --mode standard \\
    --user-id patient_123
\end{verbatim}

\textbf{DialoGPT Chatbot}:
\begin{verbatim}
python -m src.interfaces.cli.rag_chatbot_cli \\
    --mode neural \\
    --user-id patient_123
\end{verbatim}

\textbf{RAG-Enhanced Chatbot}:
\begin{verbatim}
python -m src.interfaces.cli.rag_chatbot_cli \\
    --mode enhanced \\
    --user-id patient_123 \\
    --therapy-mode
\end{verbatim}

\subsection{CLI Commands}

The enhanced mode CLI supports the following commands:

\begin{verbatim}
help     - Show all available commands
profile  - Display patient profile from MongoDB
patterns - Show behavior pattern analysis
trends   - Show trending behavior patterns (7-day)
search   - Search past conversations
emotions - View emotion history
memory   - View memory statistics
reset    - Clear conversation history
quit     - Exit the chatbot
\end{verbatim}

\subsection{Example Conversation Flows}

\textbf{AIML Example}:
\begin{verbatim}
User: I feel anxious
Bot: Anxiety can feel very overwhelming.
      What situations or thoughts are making
      you feel anxious right now?

User: I'm worried about my job
Bot: Work stress is very common. Can you tell
      me more about what specifically is
      worrying you?
\end{verbatim}

\textbf{RAG-Enhanced Example (with metadata)}:
\begin{verbatim}
User: I've been feeling really lonely lately

Bot Response: "I hear that you're experiencing
loneliness. That's a difficult feeling to carry.
Can you tell me more about what's happening
in your life right now?"

Metadata Displayed:
  📊 Detected Emotion: sadness (medium intensity)
  ⚠️  Pattern Detected: social_withdrawal (mild severity)
  🟢 Risk Assessment: LOW

[1.23s | 📥 245 | 📤 89 | 📊 334 tokens |
 📈 0.0318% context | 📚 5 memories]

💬 What specific situations are making
   you feel isolated?
\end{verbatim}

\subsection{Emotion Categories}

The EmotionDetectionService detects 12 emotion categories:

\begin{itemize}
    \item \textbf{joy}: Happiness, pleasure, satisfaction
    \item \textbf{sadness}: Grief, sorrow, unhappiness
    \item \textbf{anger}: Frustration, irritation, rage
    \item \textbf{anxiety}: Worry, nervousness, panic
    \item \textbf{fear}: Phobias, apprehension, terror
    \item \textbf{disgust}: Revulsion, aversion
    \item \textbf{surprise}: Shock, amazement
    \item \textbf{hope}: Optimism, expectation
    \item \textbf{gratitude}: Appreciation, thankfulness
    \item \textbf{loneliness}: Isolation, disconnection
    \item \textbf{frustration}: Blocked goals, irritation
    \item \textbf{confusion}: Uncertainty, puzzlement
\end{itemize}

Each detection includes:
\begin{itemize}
    \item Primary emotion
    \item Confidence score (0-1)
    \item Intensity level (low/medium/high)
    \item Sentiment (positive/negative/neutral)
    \item Keywords that triggered detection
\end{itemize}

\subsection{Behavior Pattern Categories}

The BehaviorPatternService detects 15 mental health patterns:

\begin{itemize}
    \item social\_withdrawal
    \item sleep\_disruption
    \item mood\_swings
    \item appetite\_change
    \item anxiety\_increase
    \item depression\_symptoms
    \item self\_harm\_risk (critical)
    \item substance\_use
    \item irritability
    \item concentration\_difficulty
    \item hopelessness
    \item agitation
    \item fatigue
    \item guilt
    \item avoidance
\end{itemize}

Each pattern includes:
\begin{itemize}
    \item Pattern type
    \item Confidence score
    \item Severity (mild/moderate/severe)
    \item Description
    \item Keywords/phrases that triggered detection
\end{itemize}

\subsection{Data Models}

\textbf{PatientProfile (MongoDB)}:
\begin{verbatim}
{
    "user_id": "patient_123",
    "name": "John Doe",
    "created_at": "2026-01-24T10:00:00Z",
    "demographics": {...},
    "known_triggers": ["work stress", "social isolation"],
    "treatment_goals": ["reduce anxiety", "improve sleep"],
    "emotion_summary": {...},
    "behavior_patterns": [...],
    "risk_level": "medium",
    "session_notes": [...]
}
\end{verbatim}

\textbf{Qdrant Point Payload}:
\begin{verbatim}
{
    "user_id": "patient_123",
    "session_id": "session_456",
    "role": "user",  # or "assistant"
    "text": "I'm feeling anxious today",
    "timestamp": "2026-01-24T10:05:00Z",
    "emotion": "anxiety",
    "emotion_confidence": 0.85,
    "emotion_intensity": "medium",
    "sentiment": "negative",
    "behavior_patterns": ["anxiety_increase"],
    "next_question": "What's making you feel anxious?"
}
\end{verbatim}

\subsection{Dependencies}

\textbf{Key Python Packages}:
\begin{itemize}
    \item \texttt{aiml}: Rule-based pattern matching
    \item \texttt{transformers}: DialoGPT model and tokenizer
    \item \texttt{qdrant-client}: Vector database client
    \item \texttt{pymongo}: MongoDB client
    \item \texttt{google-genai}: Gemini API client
    \item \texttt{sentence-transformers}: Sentence-BERT embeddings
    \item \texttt{nltk}: Text preprocessing
    \item \texttt{python-dotenv}: Environment configuration
\end{itemize}

\textbf{Full dependency list}: See \texttt{requirements.txt} in the repository root.

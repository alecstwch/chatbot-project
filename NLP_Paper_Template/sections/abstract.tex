\begin{abstract}
\label{abstract}

Conversational agents have become increasingly important in various domains, from mental health support to general customer service. This paper presents a comprehensive study comparing three distinct approaches to chatbot development: rule-based (AIML), neural network (DialoGPT), and hybrid (GPT-2 with intent classification). We implement and evaluate these approaches on two domains: psychotherapy conversations and general dialogue. Our hybrid model, combining zero-shot intent classification with GPT-2 generation, achieves superior performance with F1-score of 0.83 (vs. 0.69 for AIML and 0.75 for DialoGPT) and BLEU score of 0.62 (vs. 0.45 and 0.58 respectively). Through comprehensive error analysis and explainability studies using LIME, we identify that 60\% of errors stem from intent misclassification. Our work demonstrates that combining symbolic reasoning with neural generation provides a practical balance between interpretability, performance, and computational efficiency. We make all code, models, and evaluation scripts publicly available to facilitate reproducibility and future research.

\end{abstract}
